Une signature en anneau est composée d'un anneau et d'une signature. 
Un anneau est un ensemble de clefs publiques dont l'une appartient au signataire. 
La signature est générée à l'aide de cet anneau, et toute personne qui la 
vérifierait ne pourrait pas savoir quel membre de l'anneau est le véritable 
signataire.\\
Une signature en anneau d'un message $m$ avec les clefs publiques 
\{$K_1, K_2,...,K_n$\} prouve qu'une personne ayant connaissance de l'une des clefs 
privées \{$k_1, k_2,...,k_n$\} a signé le message $m$.\\
Une signature en anneau est utilisée pour anonymiser l'identité de l'expéditeur 
parmi un nombre de signataires potentiels. 
Les entrées de la transaction (\acrshort{utxo}) sont cachées dans un anneau. 
Une image clef (\textit{\og key image \fg})
associée  à une signature en anneau garantit que, même si on ne peut pas déterminer
la source d'une transaction, il est  façile de vérifier si l'expéditeur a tenté 
d'envoyer les mêmes fonds plusieurs fois (problème de la double dépense). 
La propriété d'intraçabilité\cite[1]{monero_wp} (\og \textit{untraceability}\fg) est 
satisfaite : pour chaque transaction entrante, tous les expéditeurs possibles sont 
équiprobables.\\ 
Plus la taille de l'anneau est grand, plus la propriété d'intraçabilité est forte.\\
Du point de vue de l'observateur, il peut résoudre un ensemble de clefs publiques 
à partir de la signature ($\Gamma := \sigma \mapsto \{K_1, K_2,...,K_n\}$).\\
Son objectif est de déterminer l'index $\pi$ du réel signataire parmi \{$K_1, K_2,...,K_n$\}.

\bigskip 
\noindent
$H_p$ : une fonction de hachage déterministe E($\mathds{F}_q) \rightarrow E( \mathds{F}_q)$

\bigskip 
\noindent 
Une signature en anneau contient 4 algorithmes :
\begin{itemize}
    \item GEN : le signataire génère $x \in [l-1]$, calcule la clef publique $P = xG$ 
    et l'image clef correspondante $I = xH_p(P)$\footnotemark
    \item SIG : prend un message $m$, un ensemble $S'$ de clefs publiques 
    {$P_i$}($i \neq \pi$), une paire ($P_\pi, x_\pi$) et produit une signature
    $\sigma$ et un ensemble $S=S'\cup{P_\pi}$ 
    \item VER : prend un message $m$, un ensemble S, une signature $\sigma$ et
    indique en sortie "vrai" ou "faux"
    \item LNK : prend un ensemble $\mathcal{I}=\{I_i\}$ et vérifie si $I$ est contenue dans l'ensemble : lorsqu’une nouvelle transaction est émise sur le réseau, sa signature est vérifiée (VER). Si la signature est correcte, l'image clef de la transaction est comparée aux éléments de l’ensemble, noté $\mathcal{I}$, des images clefs de toutes les transactions passées. Si l'image clef appartient à $\mathcal{I}$, cela signifie que le signataire a effectué une double dépense et la transaction est refusée. Sinon, l'image clef est ajoutée à $\mathcal{I}$ et la transaction est acceptée.
\end{itemize}

\begin{comment}
\begin{proposition}
    Soient deux agents $A_1$,$A_2$ et un ensemble d'agents 
    A ($A_1 \in A$ et $A_2 \notin A)$\\
    $A_1$ envoie un message à $A_2$ via une signature en anneau : 
    $A \rightarrow A_2$\\
    Une signature en anneau garantit un anonymat $\alpha_5$ de l'émetteur si 
    $\forall a \in A, p(a \rightarrow A_2)$ est équiprobable. 
\end{proposition}
\end{comment}

\begin{proposition}
    Une signature en anneau garantit un anonymat $\alpha_2$ de l'émetteur si
    la résolution suivante est difficile : 
    $\Gamma := A \mapsto A_1$. 
\end{proposition}

\begin{comment}
    Another downside to the original CryptoNote set-up is that it requires a given
    pair of (P, A) of pubkey P and amount A to be used in a ring signature with other
    pubkeys having the same amount. For less common amounts, this means there may
    be a smaller number of potential pairs (P ′, A′) available on the blockchain with
    A′ = A to ring signature with. Thus, in the original CryptoNote protocol, the
    potential anonymity set is perhaps smaller than may be desired. Analysis of the
    above weaknesses is covered in [9]
\end{comment}