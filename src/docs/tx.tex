Une transaction (\og tx \fg) est un objet contenant plusieurs champs dont certains champs sont 
des espaces de noms (adresse) et d'autres sont des métadonnées (horodatage).\\
Une transaction représente un mouvement de monnaie d'une adresse à une autre, 
signée avec la clef privée de l'expéditeur, qui souhaite réattribuer la possession d'une 
certaine quantité de monnaie à une adresse de destination spécifiée dans la transaction.\\
Plus précisément, une transaction consiste en :
\begin{itemize}
    \item un ensemble d'entrées (\acrshort{utxo}). Chaque entrée contient un montant associé
    \item un ensemble de sorties ou d'adresses de destination
    \item un montant à transférer à chaque sortie
\end{itemize}

\medskip
\noindent
Les transactions Bitcoin \& Monero sont définis ci-dessous respectivement.\\

\begin{table}[!ht]
    \centering
    \begin{tabular}{|l|l|l|}
    \hline
        espace & taille en bits & description \\ \hline
        hash & 256 & haché de la transaction \\ \hline
        input & 160 & adresse(s) d'entrée \\ \hline
        output & 160 & adresse(s) de sortie \\ \hline
        value & ~ & montants associés aux adresses d'entrée et de sortie \\ \hline
        signature & ~ & preuve que l'expéditeur a autorisé la transaction \\ \hline
    \end{tabular}
    \caption{Transaction Bitcoin}
\end{table}

\begin{table}[!ht]
    \centering
    \begin{tabular}{|l|l|l|}
    \hline
        espace & taille en bits & description \\ \hline
        hash & 256 & haché de la transaction \\ \hline
        input & 256 & image(s) clé(s) \\ \hline
        output & 256 & adresse(s) jetable(s) \\ \hline
        %extra & ~ & stockage de donnée arbitraire par le mineur \\ \hline
        payment id & 64 & chaîne de caractères unique permettant d'identifier une transaction \\ \hline
        ring signature &  & preuve que l'expéditeur a autorisé la transaction \\ \hline
    \end{tabular}
    \caption{Transaction Monero}
\end{table}