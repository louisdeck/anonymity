La notion de \og\textit{mix}\fg\cite{mix_wp} a été présenté par David Chaum
en 1981, dans le cadre de communications anonymes d'emails.
Au lieu d'envoyer directement un message à $A_2$, $A_1$ transmet le message 
à un ou plusieurs \textit{mixes}. Chaque \textit{mix} attend 
qu'un certain nombre de messages arrivent avant de les déchiffrer et de les
transmettre dans un ordre aléatoire, cachant ainsi la correspondance entre 
les messages entrants et sortants. Un mix est donc un appareil qui stocke 
et transmet des messages avec une latence élevée.
Cela permet d'offusquer la relation entre émetteurs et récepteurs, dans un objectif 
de préserver l'anonymat relationnelle (qui communique avec qui).

\medskip 
\noindent
Dans le cadre de la \textit{blockchain}, un mixeur est un service 
qui permet d'offusquer la relation entre entrées et sorties d'une transaction.
Ce service peut être centralisé, c'est à dire qu'il dépend d'un serveur 
central pour effectuer le mixing.
Un problème de confiance est inhérent avec cette approche : il n'y a 
aucune certitude que ce type de service ne conserve des logs 
(adresses \textit{blockchain}, adresses \acrshort{ip}..) ou redistribue 
les jetons. D'autres alternatives décentralisées existent, 
ne dépendant pas d'un serveur de mixing centralisé.

\begin{definition}[Mixeur]
    Un mixeur est un quadruplet X=(E,R,M,N) où :
    \begin{itemize}
        \item E est un ensemble fini d'émetteurs 
        \item R est un ensemble fini de récepteurs
        \item M est un ensemble fini de messages
        \item N est un ensemble fini de {\textit{mixes}}
    \end{itemize}
\end{definition}

\begin{proposition}
    Soient 4 agents $A_1, A_2, A_3, A_4$ dont $A_3$ et $A_4$ appartiennent à un mixeur.\\
    $A_1$ envoie un message à $A_2$ via un mixeur : $A_1 \rightarrow A_3 \wedge A_4 
    \rightarrow A_2$\\
    Un mixeur garantit un anonymat $\alpha_1$ si les résolutions suivantes sont difficiles 
    :\\
    $\Gamma_1 := A_3 \mapsto A_4 \wedge \Gamma_2 := A_4 \mapsto A_3$
\end{proposition}