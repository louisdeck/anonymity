\noindent
Les échanges financiers sur Internet reposent presque exclusivement via les institutions 
financières, qui agissent comme tiers de confiance pour le traitement des paiements 
électroniques.\\
L'anonymat et la décentralisation sont des caractéristiques intéressantes à explorer
pour le futur des paiements électroniques, à cause des inévitables conflits d'intérêt 
entre les autorités centrales et les utilisateurs.\\
En 2008, \og Bitcoin\cite{bitcoin_wp}: A Peer-to-Peer Electronic Cash System \fg publié 
sous le pseudonyme Satoshi Nakamoto, a partagé une solution permettant à deux parties 
d'échanger de la monnaie électronique.\\
La particularité de cette solution est la suppression de ce modèle de confiance 
par l'ajout d'une preuve cryptographique.

\bigskip

%Bitcoin a été la première solution pair à pair, pseudonyme et voulue décentralisée.\\
À ses débuts, Bitcoin a été qualifié comme une monnaie électronique \og anonyme \fg
et non pseudonyme.
Le pseudonymat de l'utilisateur repose sur l'hypothèse que son pseudonyme, 
l'adresse d'un compte dérivée à partir d'une paire de clefs, ne soit pas lié à sa 
véritable identité.
D'autres vulnérabilités peuvent être exploitées par un adversaire souhaitant désanonymiser 
des utilisateurs. Le réseau Bitcoin est un réseau \acrshort{p2p} où les participants 
sont interconnectés via un canal \acrshort{tcp} non chiffré. Les participants du réseau, plus connus 
sous le nom de noeuds, maintiennent à jour une liste d'adresses \acrshort{ip} de leurs noeuds voisins. 
Après qu'une transaction ait été créée, celle-ci est propagée aux noeuds voisins du noeud 
originaire de la transaction, qui vont eux-mêmes propager la transaction à leurs propres 
noeuds voisins. Si un noeud contrôlé par un adversaire peut s'assurer qu'une connexion entrante 
faisant part d'une transaction provient de l'auteur même de la transaction, l'adversaire peut
corréler l'adresse \acrshort{ip} du noeud et la transaction, compromettant l'anonymat.
Une atténuation est d'utiliser un \acrshort{vpn} (de confiance !) ou \acrshort{tor} afin d'offusquer sa 
réelle adresse \acrshort{ip}.
Un autre risque provient de la nature publique des transactions (voir 2.2) et de
l'exploitation du \acrshort{tga}.

\bigskip

Depuis, des protocoles comme CryptoNote\cite{monero_wp} à l'origine de Monero, ont émergé 
mettant l'accent sur la vie privée et l'anonymat. Le \acrshort{tga} n'est plus exploitable car : 
\begin{itemize}
    \item les entrées de transaction ne sont plus liées à un seul émetteur grâce 
    aux signatures en anneau 
    \item une adresse de réception unique est générée par l'émetteur pour chaque transaction grâce
    aux adresses jetables (\og one-time addresses \fg)
    \item les montants sont rendus confidentiels grâce à \acrshort{ringct}
    %RingCT est utilisé pour cacher le montant de la transaction en combinant plusieurs 
    %transactions avec le même montant de sortie en une seule transaction.
\end{itemize}

\medskip 
\noindent
Le prochain chapitre sera consacré à introduire la notion d'espaces de noms (\og \textit{namespace} \fg).
L'un des objectifs de ce travail est d'énumérer les espaces de noms en rapport à l'anonymat 
ainsi que des relations entre espaces de noms pouvant compromettre l'anonymat ou au contraire 
le garantir.