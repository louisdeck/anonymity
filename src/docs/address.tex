Dans le contexte d'une \textit{\gls{blockchain}}, une adresse est un espace de noms qui sert 
de mécanisme de routage dans un système distribué. Une adresse 
Bitcoin\cite{bitcoin_address} (\acrshort{p2pkh}) est composée de 160 bits.
Un espace d'adressage fait référence au nombre d'adresses disponibles dans un espace 
donné, il est de $2^{160}$ pour une adresse Bitcoin. 
Chaque adresse, étant la sortie du haché de la clef publique correspondante, peut être 
considérée comme une clé unique.

\bigskip 
\noindent
Les adresses Bitcoin \& Monero\cite{monero_address, monero_address2} sont définies 
ci-dessous respectivement.\\
$G$ représente un point de base de la courbe elliptique (secp256k1\cite{bitcoin_secp} 
pour Bitcoin et Ed25519\cite[5]{monero_wp} pour Monero).

\begin{table}[!ht]
    \centering
    \begin{tabular}{|l|l|l|}
    \hline
        espace & taille en bits & description \\ \hline
        priv key $k$ & 256 & génération de bits pseudo-aléatoire \\ \hline
        pub key $K$ & 256 & $K=kG$ \\ \hline
        address & 160 & base58(0x00 + ripemd160(sha256($K$)) + checksum) \\ \hline
        balance & ~ & \acrshort{utxo} \\ \hline
    \end{tabular}
    \caption{Adresse Bitcoin}
\end{table}

\begin{table}[!ht]
    \centering
    \begin{tabular}{|l|l|l|}
    \hline
        espace & taille en bits & description \\ \hline
        priv spend key $k^s$ & 256 & génération de bits pseudo-aléatoire \\ \hline
        priv view key $k^v$ & 256 & $k^v$ = $H(k^s)$ \\ \hline
        pub spend key $K^s$ & 256 & $K^s = k^sG$ \\ \hline
        pub view key $K^v$ & 256 & $K^v = k^vG$ \\ \hline
        address & 552 & base58(0x12 + $K^s||K^v$ + checksum) \\ \hline
        open alias & ~ & nom d'hôte associé à une adresse \\ \hline
        balance & ~ & \acrshort{utxo} \\ \hline
    \end{tabular}
    \caption{Adresse Monero}
\end{table}

\noindent 
La somme de contrôle d'une adresse Bitcoin est égale à : 
sha256(sha256(0x00 + ripemd160(sha256($K$))))[0:31]\\
La somme de contrôle d'une adresse Monero est égale à : 
keccak256(0x18 + $K$)[0:31]

\bigskip 

\begin{itemize}
    \item Les clefs Monero sont deux fois plus grandes que les clefs Bitcoin
    \item Les adresses Monero possèdent 392 bits de plus que les adresses Bitcoin
    \item Monero scinde les clefs en 2 usages : \og \textit{spend} \fg pour défenser 
    des fonds et \og \textit{view} \fg pour obtenir le solde et l'historique de transactions
    d'une adresse
    \item Le solde et l'historique de transactions d'une adresse Bitcoin sont publiques.
    Contrairement à Monero, où l'on peut choisir de communiquer $k_v$
    \item Dans une transaction Bitcoin, les adresses et les montants associés 
    sont publiques tandis que Monero anonymise ces informations.
\end{itemize}