\acrshort{tor}\cite{tor_wp} est un protocole implémentant un routage en oignon.
Le routage en oignon est un réseau superposé et distribué
permettant d'anonymiser les applications basées sur le protocole \acrshort{tcp}.
Chaque utilisateur (\og \textit{onion proxy}\fg) choisit un chemin à travers le réseau et 
construit un circuit, dans lequel chaque noeud (\og \textit{onion router}\fg) connaît 
uniquement son prédecesseur et son successeur dans celui-ci. 
%Certains noeuds sont de confiance et font office de \og \textit{directory servers}\fg , 
%ils fournissent les noeuds connus et leurs états.
Contrairement à un \og \textit{chaum-mix}\fg, un noeud ne stocke pas les messages 
et les transmet le plus vite possible avec une volonté de fournir 
une faible latence aux utilisateurs.

\medskip 

Le trafic transite à travers le circuit dans des cellules de taille fixe de 512 bits, 
qui sont déballées (\og \textit{unwrapped}\fg) avec une clef symétrique \acrshort{aes} 
de 128 bits à chaque nœud et relayées en aval.
Chaque noeud maintient à long terme une clef d'identité et à court terme une clef oignon.
La première est utilisée pour signer les certificats \acrshort{ssl} et une description 
des paramètres (clefs, adresses, politiques de sortie) de son noeud.
La seconde est utilisée pour déchiffrer les requêtes des utilisateurs souhaitant 
établir un circuit et négocier des clefs éphémères.
Les noeuds communiquent donc entre eux et avec les utilisateurs via des connexions 
\acrshort{ssl} avec ces clefs éphémères.

\medskip 

Un client construit un circuit progressivement, négociant une clef symétrique
avec chaque noeud du circuit, un \og \textit{hop}\fg à la fois.
Pour commencer à créer un circuit, le client envoie une cellule 
\og \textit{create}\fg au premier noeud dans son chemin choisi. 
La charge utile de cette cellule contient la première moitié de la 
poignée de main Diffie-Hellman ($g^x$), chiffrée avec la clef oignon
du noeud. Le noeud répond avec une cellule \og \textit{created}\fg
contenant $g^y$ et un haché de la clef négociée $K=g^{xy}$.
Le client n'utilise pas de clef publique et reste anonyme! Le client et
le noeud se mettent d'accord sur une clef et Alice sait uniquement 
que le noeud l'a apprise.

\begin{definition}[Arrangement]
    Un arrangement, noté $A^k_n$ est une permutation de k élements parmi n éléments 
    ($k \leq n$).
    Les éléments sont pris sans répétition et sont ordonnés.   
\end{definition}

\begin{definition}[Circuit]
    Soient un ensemble fini $A$ d'agents, un ensemble fini $N$ de \textit{mixes}, 
    un ensemble fini $C$ de circuits, un ensemble fini $K_s$ de clefs symétriques et un 
    ensemble fini $K_a$ de clefs asymétriques.

    \medskip
    \noindent
    Un \textit{mix} $n \in N$ est un tuple \{$ip\_addr_{flag}$, tcp\_port, $K_s^1$,
    $K_a^1$\} dont flag = entry $\vee$ middle $\vee$ exit\\
    $K_s^1 \in K_s, K_a^1 \in K_a$

    \medskip 
    \noindent
    Un utilisateur $u \in U$ est un tuple \{ip\_addr, tcp\_port, $K_a$\}.

    \medskip 
    \noindent
    Un circuit $c \in C$ est un arrangement de \textit{mixes} pour $a \in A$.
\end{definition}

\begin{proposition}
    Un routage en oignon garantit un anonymat $\alpha_2$ si les résolutions suivantes 
    sont difficiles :\\ 
    $\forall c \in C, \: \Gamma := ip\_addr_{exit}^n \mapsto ip\_addr^u$
\end{proposition}