Soit un ensemble d'agents $(A_1,...,A_n)$ qui s'échangent des messages sur un canal 
de communication par défaut unidirectionnel (et bidirectionnel lorsque précisé). 
Un agent peut être un émetteur ou/et un récepteur. Tout message transite vers un système d'anonymat
dans un modèle de boîte noire.
Selon la nature de la méthode d'anonymat utilisée, un agent peut être dénoté par 
une adresse \acrshort{ip}, une clef publique..
Le canal de communication est surveillé par un observateur, qui souhaite par exemple 
connaître l'identité de l'émetteur ou du récepteur d'un message.
Cet observateur est un adversaire honnête mais curieux\cite[2]{hbc}, un participant légitime qui ne s'écarte pas 
du protocole défini mais tente d'obtenir toutes les informations possibles à partir 
des messages reçus légitimement. Les \textit{\gls{blockchain}s} Bitcoin et Monero étant 
publiques, l'observateur a accès possiblement en lecture à toutes les transactions 
publiées sur ces registres. On s'intéressera principalement à l'anonymat de l'émetteur, 
le principe est analogue pour l'anonymat du récepteur. L'anonymat de l'émetteur 
consiste à être indistinguable\cite{anon_term} au sein d'un ensemble, l'ensemble d'anonymat 
(\og \textit{anonymity set}\fg). Il y a un ensemble d'émetteurs possibles
et l'observateur ne peut pas faire la distinction entre les émetteurs de cet 
ensemble. L'observateur souhaite connaître quel émetteur parmi l'ensemble d'anonymat 
est à l'origine d'un message, il assigne une probabilité pour chaque émetteur. Deux méthodes d'anonymat sont analysées : les adresses jetables et les signatures en anneau.\\
D'autres méthodes auraient pu être explorées :
\begin{itemize}
    \item Mixeur : service qui permet d'offusquer la relation entre entrées et sorties d'une transaction, et préserver l'anonymat relationnel. Ce service peut être centralisé, c'est à dire qu'il dépend d'un serveur central pour effectuer le mixing.
    Un problème de confiance est inhérent avec cette approche : il n'y a aucune certitude que ce type de service ne conserve des logs (adresses \textit{blockchain}, adresses \acrshort{ip}..) ou redistribue les jetons. 
    \item \acrshort{tor} (le routage en oignon) : réseau superposé et distribué permettant d'anonymiser les applications basées sur le protocole \acrshort{tcp}.
    Chaque utilisateur (\og \textit{onion proxy}\fg) choisit un chemin à travers le réseau et construit un circuit, dans lequel chaque noeud (\og \textit{onion router}\fg) connaît uniquement son prédecesseur et son successeur dans celui-ci.
    \item \acrshort{vpn} IPsec mode tunnel
\end{itemize}

\begin{comment}
\subsection{Mixeur}
La notion de \og\textit{mix}\fg\cite{mix_wp} a été présenté par David Chaum
en 1981, dans le cadre de communications anonymes d'emails.
Au lieu d'envoyer directement un message à $A_2$, $A_1$ transmet le message 
à un ou plusieurs \textit{mixes}. Chaque \textit{mix} attend 
qu'un certain nombre de messages arrivent avant de les déchiffrer et de les
transmettre dans un ordre aléatoire, cachant ainsi la correspondance entre 
les messages entrants et sortants. Un mix est donc un appareil qui stocke 
et transmet des messages avec une latence élevée.
Cela permet d'offusquer la relation entre émetteurs et récepteurs, dans un objectif 
de préserver l'anonymat relationnelle (qui communique avec qui).

\medskip 
\noindent
Dans le cadre de la \textit{blockchain}, un mixeur est un service 
qui permet d'offusquer la relation entre entrées et sorties d'une transaction.
Ce service peut être centralisé, c'est à dire qu'il dépend d'un serveur 
central pour effectuer le mixing.
Un problème de confiance est inhérent avec cette approche : il n'y a 
aucune certitude que ce type de service ne conserve des logs 
(adresses \textit{blockchain}, adresses \acrshort{ip}..) ou redistribue 
les jetons. D'autres alternatives décentralisées existent, 
ne dépendant pas d'un serveur de mixing centralisé.

\begin{definition}[Mixeur]
    Un mixeur est un quadruplet X=(E,R,M,N) où :
    \begin{itemize}
        \item E est un ensemble fini d'émetteurs 
        \item R est un ensemble fini de récepteurs
        \item M est un ensemble fini de messages
        \item N est un ensemble fini de {\textit{mixes}}
    \end{itemize}
\end{definition}

\begin{proposition}
    Soient 4 agents $A_1, A_2, A_3, A_4$ dont $A_3$ et $A_4$ appartiennent à un mixeur.\\
    $A_1$ envoie un message à $A_2$ via un mixeur : $A_1 \rightarrow A_3 \wedge A_4 
    \rightarrow A_2$\\
    Un mixeur garantit un anonymat $\alpha_1$ si les résolutions suivantes sont difficiles 
    :\\
    $\Gamma_1 := A_3 \mapsto A_4 \wedge \Gamma_2 := A_4 \mapsto A_3$
\end{proposition}

\subsection{Le routeur en oignon (\acrshort{tor})}
\acrshort{tor}\cite{tor_wp} est un protocole implémentant un routage en oignon.
Le routage en oignon est un réseau superposé et distribué
permettant d'anonymiser les applications basées sur le protocole \acrshort{tcp}.
Chaque utilisateur (\og \textit{onion proxy}\fg) choisit un chemin à travers le réseau et 
construit un circuit, dans lequel chaque noeud (\og \textit{onion router}\fg) connaît 
uniquement son prédecesseur et son successeur dans celui-ci. 
%Certains noeuds sont de confiance et font office de \og \textit{directory servers}\fg , 
%ils fournissent les noeuds connus et leurs états.
Contrairement à un \og \textit{chaum-mix}\fg, un noeud ne stocke pas les messages 
et les transmet le plus vite possible avec une volonté de fournir 
une faible latence aux utilisateurs.

\medskip 

Le trafic transite à travers le circuit dans des cellules de taille fixe de 512 bits, 
qui sont déballées (\og \textit{unwrapped}\fg) avec une clef symétrique \acrshort{aes} 
de 128 bits à chaque nœud et relayées en aval.
Chaque noeud maintient à long terme une clef d'identité et à court terme une clef oignon.
La première est utilisée pour signer les certificats \acrshort{ssl} et une description 
des paramètres (clefs, adresses, politiques de sortie) de son noeud.
La seconde est utilisée pour déchiffrer les requêtes des utilisateurs souhaitant 
établir un circuit et négocier des clefs éphémères.
Les noeuds communiquent donc entre eux et avec les utilisateurs via des connexions 
\acrshort{ssl} avec ces clefs éphémères.

\medskip 

Un client construit un circuit progressivement, négociant une clef symétrique
avec chaque noeud du circuit, un \og \textit{hop}\fg à la fois.
Pour commencer à créer un circuit, le client envoie une cellule 
\og \textit{create}\fg au premier noeud dans son chemin choisi. 
La charge utile de cette cellule contient la première moitié de la 
poignée de main Diffie-Hellman ($g^x$), chiffrée avec la clef oignon
du noeud. Le noeud répond avec une cellule \og \textit{created}\fg
contenant $g^y$ et un haché de la clef négociée $K=g^{xy}$.
Le client n'utilise pas de clef publique et reste anonyme! Le client et
le noeud se mettent d'accord sur une clef et Alice sait uniquement 
que le noeud l'a apprise.

\begin{definition}[Arrangement]
    Un arrangement, noté $A^k_n$ est une permutation de k élements parmi n éléments 
    ($k \leq n$).
    Les éléments sont pris sans répétition et sont ordonnés.   
\end{definition}

\begin{definition}[Circuit]
    Soient un ensemble fini $A$ d'agents, un ensemble fini $N$ de \textit{mixes}, 
    un ensemble fini $C$ de circuits, un ensemble fini $K_s$ de clefs symétriques et un 
    ensemble fini $K_a$ de clefs asymétriques.

    \medskip
    \noindent
    Un \textit{mix} $n \in N$ est un tuple \{$ip\_addr_{flag}$, tcp\_port, $K_s^1$,
    $K_a^1$\} dont flag = entry $\vee$ middle $\vee$ exit\\
    $K_s^1 \in K_s, K_a^1 \in K_a$

    \medskip 
    \noindent
    Un utilisateur $u \in U$ est un tuple \{ip\_addr, tcp\_port, $K_a$\}.

    \medskip 
    \noindent
    Un circuit $c \in C$ est un arrangement de \textit{mixes} pour $a \in A$.
\end{definition}

\begin{proposition}
    Un routage en oignon garantit un anonymat $\alpha_2$ si les résolutions suivantes 
    sont difficiles :\\ 
    $\forall c \in C, \: \Gamma := ip\_addr_{exit}^n \mapsto ip\_addr^u$
\end{proposition}

\subsection{Réseau virtuel privé (\acrshort{vpn})}
 
Un \acrshort{vpn} permet de sécuriser la couche \acrshort{tcp}/\acrshort{ip}. 
Plus précisément, un \acrshort{vpn} \acrshort{ssl} sécurise la couche 
\acrshort{tcp} tandis qu'un \acrshort{vpn} \acrshort{ipsec} sécurise la couche \acrshort{ip}.\\ 
Un \acrshort{vpn} \acrshort{ssl} crée une connexion \acrshort{tcp} entre un client et un serveur 
sécurisée par le protocole \acrshort{ssl}.\\ 
Cependant, les adresses \acrshort{ip} du client et du serveur ne sont pas offusquées car la 
couche \acrshort{ip} n'est pas prise en compte. Un \acrshort{vpn} \acrshort{ipsec} sécurise lui 
la couche \acrshort{ip}, il existe 2 modes : transport et tunnel. Le mode transport ne protège 
que la charge utile du paquet \acrshort{ip}, et non l'en-tête. Le mode tunnel protège à la fois la charge 
utile et l'en-tête du paquet \acrshort{ip}. Une nouvelle en-tête \acrshort{ip} est créée, l'en-tête
\acrshort{ip} originale est chiffrée.\\
Parmi ces 3 techniques, l'en-tête \acrshort{ip} originale est uniquement modifiée par
un \acrshort{vpn} \acrshort{ipsec} en mode tunnel.

\begin{figure}[h]
    \centering
    \includegraphics[scale=0.8]{pics/ipsec.png}
    \caption{Datagramme \acrshort{ipsec} mode \acrshort{esp}-tunnel}
\end{figure}

\begin{proposition}
    Un \acrshort{vpn} \acrshort{ipsec} en mode tunnel garantit un anonymat $\alpha_3$ 
    si la résolution suivante est difficile :\\ 
    $\Gamma \: := ip\_addr_{new} \mapsto ip\_addr_{original}$
\end{proposition}  
\end{comment}

\subsection{Adresse jetable}
Les adresses jetables sont des adresses pseudo-aléatoires à usage unique, générées 
par l'expéditeur pour chaque transaction, permettant de masquer l'adresse du 
destinataire. Le destinataire publie une seule adresse mais tous ses paiements 
entrants sont adressés à des adresses uniques, elles ne peuvent pas être liées à 
l'adresse du destinataire. La propriété de non-liaison\cite[1]{monero_wp} 
(\og \textit{unlinkability}\fg) est satisfaite : pour deux transactions sortantes, 
il est impossible de prouver qu’elles ont été envoyées à la même adresse.

\bigskip 
\noindent 
$H_s$ : une fonction de hachage cryptographique $\{0,1\}^* \rightarrow \mathds{F}_q$ \\
$G$ : Point de base de la courbe Ed25519\\
$l$ = $2^{252} + 27742317777372353535851937790883648493$
  
\begin{enumerate}
    \item Alice génère $r \in [1,l-1]$
    \item Alice calcule l'adresse jetable correspondante : $S = H_s(rK^v_b)G + K^s_b$
    \item Alice envoie une transaction avec S comme destinataire (contenant R=rG)
    \item Bob vérifie toutes les transactions qui transitent sur le réseau et calcule : 
    $S'=H_s(Rk^v_b)G + K^s_b$
    \item Bob consomme la sortie si $S = S' \: (R \: k^v_b= rG \: k^v_b = rK^v_b)$
    \item Bob peut dépenser cette sortie en signant la transaction avec $x=H_s(Rk^v_b) + k^s_b$ 
\end{enumerate}

Lorsque Bob vérifie qu'une transaction lui appartient, il effectue 2 multiplations et une addition
sur la courbe elliptique par sortie. Du point de vue de l'observateur, il n'a connaissance ni 
de $r$, ni de $k^v_b$ et il n'est pas en capacité de résoudre le
problème du logarithme discret appliqué aux courbes elliptiques (\acrshort{ecdlp}).\\
Sous ces hypothèses, l'observateur est en incapacité de connaître le récepteur du message et 
de résoudre l'adresse de Bob à partir de l'adresse jetable générée par Alice.

\begin{proposition}
    Soient 2 agents $A_1$ et $A_2$, $A_1$ envoie un message à $A_2$ via une adresse jetable: 
    $A_1 \rightarrow S$\\
    Une adresse jetable garantit un anonymat $\alpha_1$ du récepteur si
    la résolution suivante est difficile :
    $\Gamma := S \mapsto A_2$ 
\end{proposition}





\subsection{Signature en anneau}
Une signature en anneau est composée d'un anneau et d'une signature. 
Un anneau est un ensemble de clefs publiques dont l'une appartient au signataire. 
La signature est générée à l'aide de cet anneau, et toute personne qui la 
vérifierait ne pourrait pas savoir quel membre de l'anneau est le véritable 
signataire.\\
Une signature en anneau d'un message $m$ avec les clefs publiques 
\{$K_1, K_2,...,K_n$\} prouve qu'une personne ayant connaissance de l'une des clefs 
privées \{$k_1, k_2,...,k_n$\} a signé le message $m$.\\
Une signature en anneau est utilisée pour anonymiser l'identité de l'expéditeur 
parmi un nombre de signataires potentiels. 
Les entrées de la transaction (\acrshort{utxo}) sont cachées dans un anneau. 
Une image clef (\textit{\og key image \fg})
associée  à une signature en anneau garantit que, même si on ne peut pas déterminer
la source d'une transaction, il est  façile de vérifier si l'expéditeur a tenté 
d'envoyer les mêmes fonds plusieurs fois (problème de la double dépense). 
La propriété d'intraçabilité\cite[1]{monero_wp} (\og \textit{untraceability}\fg) est 
satisfaite : pour chaque transaction entrante, tous les expéditeurs possibles sont 
équiprobables.\\ 
Plus la taille de l'anneau est grand, plus la propriété d'intraçabilité est forte.\\
Du point de vue de l'observateur, il peut résoudre un ensemble de clefs publiques 
à partir de la signature ($\Gamma := \sigma \mapsto \{K_1, K_2,...,K_n\}$).\\
Son objectif est de déterminer l'index $\pi$ du réel signataire parmi \{$K_1, K_2,...,K_n$\}.

\bigskip 
\noindent
$H_p$ : une fonction de hachage déterministe E($\mathds{F}_q) \rightarrow E( \mathds{F}_q)$

\bigskip 
\noindent 
Une signature en anneau contient 4 algorithmes :
\begin{itemize}
    \item GEN : le signataire génère $x \in [l-1]$, calcule la clef publique $P = xG$ 
    et l'image clef correspondante $I = xH_p(P)$\footnotemark
    \item SIG : prend un message $m$, un ensemble $S'$ de clefs publiques 
    {$P_i$}($i \neq \pi$), une paire ($P_\pi, x_\pi$) et produit une signature
    $\sigma$ et un ensemble $S=S'\cup{P_\pi}$ 
    \item VER : prend un message $m$, un ensemble S, une signature $\sigma$ et
    indique en sortie "vrai" ou "faux"
    \item LNK : prend un ensemble $\mathcal{I}=\{I_i\}$ et vérifie si $I$ est contenue dans l'ensemble : lorsqu’une nouvelle transaction est émise sur le réseau, sa signature est vérifiée (VER). Si la signature est correcte, l'image clef de la transaction est comparée aux éléments de l’ensemble, noté $\mathcal{I}$, des images clefs de toutes les transactions passées. Si l'image clef appartient à $\mathcal{I}$, cela signifie que le signataire a effectué une double dépense et la transaction est refusée. Sinon, l'image clef est ajoutée à $\mathcal{I}$ et la transaction est acceptée.
\end{itemize}

\begin{comment}
\begin{proposition}
    Soient deux agents $A_1$,$A_2$ et un ensemble d'agents 
    A ($A_1 \in A$ et $A_2 \notin A)$\\
    $A_1$ envoie un message à $A_2$ via une signature en anneau : 
    $A \rightarrow A_2$\\
    Une signature en anneau garantit un anonymat $\alpha_5$ de l'émetteur si 
    $\forall a \in A, p(a \rightarrow A_2)$ est équiprobable. 
\end{proposition}
\end{comment}

\begin{proposition}
    Une signature en anneau garantit un anonymat $\alpha_2$ de l'émetteur si
    la résolution suivante est difficile : 
    $\Gamma := A \mapsto A_1$. 
\end{proposition}

\begin{comment}
    Another downside to the original CryptoNote set-up is that it requires a given
    pair of (P, A) of pubkey P and amount A to be used in a ring signature with other
    pubkeys having the same amount. For less common amounts, this means there may
    be a smaller number of potential pairs (P ′, A′) available on the blockchain with
    A′ = A to ring signature with. Thus, in the original CryptoNote protocol, the
    potential anonymity set is perhaps smaller than may be desired. Analysis of the
    above weaknesses is covered in [9]
\end{comment}